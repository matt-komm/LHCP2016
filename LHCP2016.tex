\documentclass{PoS}

\title{Single top cross section and properties measurements in CMS}

\ShortTitle{Single top cross section and properties measurements in CMS}

\author{
    \speaker{Matthias Komm}, on behalf of the CMS collaboration\\
    Universite Catholique de Louvain (UCL) (BE)\\
    E-mail: \email{Matthias.Komm@cern.ch}
}


\abstract{TODO rephrase: At the LHC, single top quarks are predominately produced via the $t$-channel. Measuring the properties of the production process provides a crucial probe of the theory of electroweak interactions. This paper reviews recent results on cross section measurements and coupling structure studies in pp collisions by the ATLAS and CMS collaborations at center-of-mass energies of 7, 8, and 13~TeV.}

\FullConference{
    Fourth Annual Large Hadron Collider Physics\\
    13-18 June 2016\\
    Lund, Sweden
}


\begin{document}

\section{Introduction}
Studies of single top quark production are an unique test suite of electroweak interactions involving heavy quarks. Measurements of the cross sections of the three production modes, $t$~channel, tW~channel, and $s$~channel, offer a model-independent probe of the CKM matrix element $\mathrm{V}_\mathrm{tb}$. On the other hand, searches for alternative production mechanisms as predicted in various new physics models are performed as well. Such scenarios can be charaterized in an effective expansion of the Standard Model (SM) that reflect the low energy approximation of new physics at a higher energy scale. Especially, the fact that the top quark does not hadronize before its electroweak decay allows the analysis of the predicted parity-violating coupling structure.

In this note, recent measurements and searches of single top quark production and properties by the CMS collaboration are reviewed.


\section{Cross section measurement of $t$ channel at 13~TeV}
\section{Cross section measurement of $s$ channel at 7~and 8~TeV}
\section{Single top quark polarization}
\section{Search for flavour changing neutral currents involing photons}
\section{Conclusion}

\begin{thebibliography}{99}
\bibitem{...}

\end{thebibliography}

\end{document}
